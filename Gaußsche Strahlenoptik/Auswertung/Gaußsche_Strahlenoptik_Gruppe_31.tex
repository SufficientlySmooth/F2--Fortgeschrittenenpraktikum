\documentclass[11pt,a4paper,oneside]{scrartcl}
\usepackage{requiredPackages}
\usepackage{subfig}
\interfootnotelinepenalty=10000

\def\UrlBreaks{\do/\do-\do_}
\begin{document}
\begin{titlepage}
	\centering
	{\scshape\LARGE Ludwig-Maximilians-Universität \linebreak München \par}
	\vspace{1cm}
	{\scshape\Large Fortgeschrittenenpraktikum II \par Wintersemester 22/23 \par}
	\vspace{1.5cm}
	{\huge\bfseries \par  Gaußsche Strahlenoptik\par}
	\vspace{2cm}
	{\Large\itshape Guido Osterwinter und Jan-Philipp Christ \par}
	\vfill
	{\large München, den \today\par}
\end{titlepage}

\tableofcontents
\newpage
\section{Zielsetzung}
\section{Versuchsdurchführung}
\subsection{Untersuchung eines Gaußschen Laserstrahls}
\subsection{Optischer Resonator}
\subsubsection{Aufbau des Resonators}
Damit sich im Fabry-Perot-Resonator aus sphärischen Spiegeln eine stehende Welle bilden kann, soll die Waist des Gauß-Strahls mittig zwischen den beiden Spiegeln liegen. Betrachtet man den halbdurchlässigen Spiegel, durch den der Gaußstrahl in den Resonator einfällt, als dicke Linse, so ergibt sich die folgende Transfermatrix:
\section{Ergebnisse und Diskussion}
\section{Zusammenfassung}
\newpage
\appendix
\section{Python-Skripte zur Auswertung}
%\large\textbf{Code 1 zu Plot TV5:}
%\lstinputlisting{TV5.py}\newpage
%\large\textbf{Code 2 zu Plot TV5:}
%\lstinputlisting{TV5_2.py}\newpage

\end{document}