\documentclass[11pt,a4paper,oneside]{scrartcl}
\usepackage{requiredPackages}
\usepackage{subfig}
\usepackage{cancel}
\usepackage[labelfont=bf]{caption}
\usepackage{booktabs}
\interfootnotelinepenalty=10000

\def\UrlBreaks{\do/\do-\do_}
\begin{document}
\begin{titlepage}
	\centering
	{\scshape\LARGE Ludwig-Maximilians-Universität \linebreak München \par}
	\vspace{1cm}
	{\scshape\Large Fortgeschrittenenpraktikum II \par Wintersemester 22/23 \par}
	\vspace{1.5cm}
	{\huge\bfseries \par  Rheologie\par}
	\vspace{2cm}
	{\Large\itshape Guido Osterwinter und Jan-Philipp Christ \par}
	\vfill
	{\large München, den \today\par}
\end{titlepage}

\tableofcontents
\newpage
\section{Zielsetzung}

\section{Theoretischer Hintergrund}
\subsection{Gaußstrahlen}
\subsubsection{Ideale Gaußstrahlen}

\subsubsection{Reale Gaußstrahlen}\label{Reale Gaußstrahlen}

\subsubsection{Beispiel für Anwendung des 'ABCD'-Gesetzes}

\subsection{Optische Resonatoren}

\section{Versuchsdurchführung}
\subsection{Bestimmung der Laserleistung und der Hintergrundhelligkeit}

\subsection{Untersuchung eines Gaußschen Laserstrahls}

\subsubsection{Vermessung der Strahltaille}\label{Durchführung Vermessung der Strahltaille}

\subsection{Optischer Resonator}

\subsubsection{Justieren und Bestimmen der Transmissionsfunktion}

\section{Ergebnisse und Diskussion}
\subsection{Bestimmung der Laserleistung und der Hintergrundhelligkeit}

\subsection{Einkoppeln des Laserstrahls}
\subsection{Untersuchung eines Gaußschen Laserstrahls}
\subsubsection{Vermessung der Strahltaille}\label{Auswertung Vermessung der Strahltaille}

\subsection{Optischer Resonator}
\subsubsection{Bestimmung der Finesse}

\section{Zusammenfassung}
\newpage

\bibliography{literatur} 
\bibliographystyle{ieeetr}
\appendix


\section{Plots zur Vermessung des Strahldurchmessers}
\subsection{Ohne Linse (kollimiert)}
%\plot{coll_z_75}{axiales Strahlprofil bei $z=75\mathrm{mm}$}{.8}{Messdaten Errorfunktion/Collimated_z_75.pdf}

\section{Python-Skripte zur Auswertung}
\subsection{Bestimmung des axialen Strahlprofils}
%\large\textbf{Code 1 zu Plot TV5:}
%\lstinputlisting{Messdaten Errorfunktion/Erf_plot.py}
\end{document}