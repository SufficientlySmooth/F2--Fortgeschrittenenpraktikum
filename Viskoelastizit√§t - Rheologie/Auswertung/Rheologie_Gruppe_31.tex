\documentclass[11pt,a4paper,oneside]{scrartcl}
\usepackage{requiredPackages}
\usepackage{subfig}
\usepackage{cancel}
\usepackage[labelfont=bf]{caption}
\usepackage{booktabs}
\interfootnotelinepenalty=10000

\def\UrlBreaks{\do/\do-\do_}
\begin{document}
\begin{titlepage}
	\centering
	{\scshape\LARGE Ludwig-Maximilians-Universität \linebreak München \par}
	\vspace{1cm}
	{\scshape\Large Fortgeschrittenenpraktikum II \par Wintersemester 22/23 \par}
	\vspace{1.5cm}
	{\huge\bfseries \par  Rheologie\par}
	\vspace{2cm}
	{\Large\itshape Guido Osterwinter und Jan-Philipp Christ \par}
	\vfill
	{\large München, den \today\par}
\end{titlepage}

\tableofcontents
\newpage
\section{Zielsetzung und Motivation}

\section{Theoretischer Hintergrund}
\subsection{Elastizität und Viskosität}
\subsection{Klassifizierung von Flüssigkeiten anhand ihres Fließverhaltens}
Fließgesetz nach Ostwald und de Waele ->\cite{dewiki:192899581}
\subsection{Viskoelastizität}
\subsection{Rotationsrheometer}
\section{Versuchsdurchführung}
\subsection{Wasser-Saccharose}
\subsubsection{Anmischen der Lösungen}
\subsubsection{Scherratenmessungen}
\subsection{Wasser-Guaran}
\subsubsection{Anmischen der Lösungen}
\subsubsection{Scherratenmessungen}
\subsubsection{Frequenzversuch}
\section{Ergebnisse und Diskussion}
\subsection{Wasser-Saccharose}
\subsubsection{Scherratenmessungen}
\subsubsection{Fehlerbetrachtung}
\subsection{Wasser-Guaran}
\subsubsection{Scherratenmessungen}
\subsubsection{Frequenzversuch}
\cite{versuchsanleitung}
\cite{Stadler2014}
\cite{Hellström_2015}
\cite{Arian2021}
"The significance of under-filling can also be demonstrated by
noting that a $100\ \mathrm{\mu m}$ change in the radius of the sample (corresponding to a 0.2\% variation) will cause a 1.6\% error in the
apparent viscosity for a parallel plate setup, and 1.2\% for a
cone plate setup." (\cite{Hellström_2015})

\section{Zusammenfassung}
\newpage

\bibliography{literatur} 
\bibliographystyle{ieeetr}
\appendix


%\section{Plots zur Vermessung des Strahldurchmessers}
%\subsection{Ohne Linse (kollimiert)}
%\plot{coll_z_75}{axiales Strahlprofil bei $z=75\mathrm{mm}$}{.8}{Messdaten Errorfunktion/Collimated_z_75.pdf}

\section{Python-Skripte zur Auswertung}
\subsection{Bestimmung des Potenzgesetzes}
%\large\textbf{Code 1 zu Plot TV5:}
%\lstinputlisting{Messdaten Errorfunktion/Erf_plot.py}
\end{document}